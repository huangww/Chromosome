\documentclass[aps,showpacs,twocolumn,floatfix,prx,superscriptaddress]{revtex4-1}
\usepackage{graphicx}
\usepackage{amsfonts}
\usepackage{amsmath}
\usepackage{amssymb}
\usepackage{upgreek}
\usepackage[usenames,dvipsnames]{color}

%\bibliographystyle{apsrev}


\def\s{\sigma}
\begin{document}

\title{Polymer Maps to Particle Diffusion and Back in 1D}

\author{Wenwen Huang}
\author{Yen Ting Lin}
\author{Daniela Fr\"{o}mberg}
\author{Jaeoh Shin}
\author{Frank J\"{u}licher}
\author{Vasily Zaburdaev}
\affiliation{Max Planck Institute for the Physics of Complex Systems, N\"{o}thnitzer Str. 38, D-01187 Dresden, Germany}


\begin{abstract}
{}
\end{abstract}
\maketitle


\section{Introduction}
Many biological processes can be modelled by idealized physical concepts and
qualitatively studied through the law of physics and methods of mathematics.
A good example is the movement of DNA. Polymer models are often utilized to
describe DNA\cite{}, characterized the long chain property of its chemical
structure. 
In our study of chromosome alignment in meiotic fission yeast, a freely jointed
bead rod ring model is adopted\cite{}. Chromosome movements during the stage of
horsetail oscillation of nucleus are translated to pinned polymer loop in an
external field\cite{}. To understand to statistics of distance between loci, we
first formulate the problem in 1D, i.e. rods can only pointing to right or
left. Amazingly, we found this simple model can maps to a 1D particle diffusion
problem, well known as single file diffusion (SFD)\cite{}. On the other hand,
SFD is a paradigmatic model in non-equilibrium statistics with many
applications\cite{}. 

In this paper, we will show how to map from a pinned polymer model describing
chromosomes to a single file particle diffusion. The equivalence between polymer
and particle means that we solve one case in one picture automatically solves the
correspondence in the other. We thus show a example that the pinned polymer loop
in constant external field can be solved analytically in 1D. We demonstrate that
the famous Fermi-Dirac statistics serves as an asymptotic approximation of
statistics of rod orientation. Exact solution is also feasible by solving the
fermion number partition problem. Thus the particle picture correspondence,
which is SFD in external force field with reflecting boundary condition, is
also solved and to our knowledge is solved for the first time. Results are
verified by numerical simulations. To further demonstrate the power of this
mapping method, we also discuss the of some other applications. In particular,
the dynamics of polymer is discussed by mapping back from SFD dynamics. 

The next section we will describe how to build the mapping from polymer and
particle. In section III, some of the applications are discussed.
Finally, we give our conclusion remarks and outlook in section IV. 


\section{Statistics of one-dimensional pinned polymer loop}
\subsection{Brownian bridge}
\subsection{Fermi-Dirac statistics of rod orientations}
\subsection{Fermion integer number partition theory}
\subsection{From rods to a polymer, beauty of Gaussian statistics}

\section{Asymmetric Exclusion Process}

\section{Towards understanding dynamics}

\section{Conclusions}

\begin{acknowledgments}
We would like to acknowledge stimulating discussions with M. Majumdar.\end{acknowledgments}


\begin{thebibliography}{10}


\bibitem{alberts2002}
B.~Alberts,
  A.~Johnson,
  J.~Lewis,
  M.~Raff,
  K.~Roberts, and
  P.~Walter,
  \emph{Molecular Biology of the Cell, Fourth Edition}
  (Garland Science, New York, 2002),
  4th ed.

\bibitem{gerton2005}
J.~L. Gerton and
  R.~S. Hawley,
  Nat. Rev. Genet. \textbf{6},
  477 (2005).

\bibitem{villeneuve2001}
A.~M. Villeneuve
  and K.~J.
  Hillers, Cell
  \textbf{106}, 647 (2001).

\bibitem{McKee2004}
B.~D. McKee,
  BBA-Gene. Struct. Expr. \textbf{1677}, 165 
  (2004).

\bibitem{Egel2004}
R.~Egel,
  \emph{The Molecular Biology of Schizosaccharomyces pombe:
  Genetics, Genomics and Beyond}(Springer, Berlin-Heidelberg,
  2004).

\bibitem{davis2001}
L.~Davis and
  G.~R. Smith,
  Proc. Natl. Acad. Sci. U. S. A.
  \textbf{98}, 8395 (2001).

\bibitem{munz1994}
P.~Munz,
  Genetics \textbf{137},
  701 (1994).

\bibitem{wells2006}
J.~L. Wells,
  D.~W. Pryce, and
  R.~J. McFarlane,
  Yeast \textbf{23}, 977
  (2006).
  
 \bibitem{SM} See Supplemental Material at ... for detailed description of experimental procedures, numerical simulations, and the discussion of 3D fluctuations of the polymer loop.

\bibitem{ding1998oscillatory}
D.-Q. Ding,
  Y.~Chikashige,
  T.~Haraguchi,
  and Y.~Hiraoka,
  J. Cell Sci. \textbf{111},
  701 (1998).

\bibitem{vogel2009self}
S.~K. Vogel,
  N.~Pavin,
  N.~Maghelli,
  F.~J\"ulicher,
  and I.~M.
  Toli\'c-N\o rrelykke, PLoS Biol.
  \textbf{7}, e1000087
  (2009).

\bibitem{yamamoto2001dynamic}
A.~Yamamoto,
  C.~Tsutsumi,
  H.~Kojima,
  K.~Oiwa, and
  Y.~Hiraoka,
  Mol. Biol. Cell \textbf{12},
  3933 (2001).

\bibitem{yamamoto1999cytoplasmic}
A.~Yamamoto,
  R.~R. West,
  J.~R. McIntosh,
  and Y.~Hiraoka,
  J. Cell Biol.
  \textbf{145}, 1233 (1999).

\bibitem{ananthanarayanan2013dynein}
V.~Ananthanarayanan,
  M.~Schattat,
  S.~K. Vogel,
  A.~Krull,
  N.~Pavin, and
  I.~M. Toli\'c-N\o rrelykke,
  Cell \textbf{153}, 1526
  (2013).

\bibitem{koszul2009dynamic}
R.~Koszul and
  N.~Kleckner,
  Trends Cell Biol. \textbf{19},
  716 (2009).

\bibitem{ding2004dynamics}
D.-Q. Ding,
  A.~Yamamoto,
  T.~Haraguchi,
  and Y.~Hiraoka,
  Dev. Cell \textbf{6},
  329 (2004).

\bibitem{wynne2012dynein}
D.~J. Wynne,
  O.~Rog,
  P.~M. Carlton,
  and A.~F.
  Dernburg, J. Cell Biol.
  \textbf{196}, 47 (2012).

\bibitem{Doi1986}
M.~Doi and
  S.~Edwards,
  \emph{The theory of polymer dynamics, International series
  of monographs on physics} (Clarendon Press, Oxford,
  1986).
  
\bibitem{foot0}
{\color{blue} Since the recombination machinery, locally altering the properties of the chromatin, becomes active only after the homologous chromosomes come to close proximity, we assume the effective temperature to be spatially uniform.}

\bibitem{Revuz1999}
D.~Revuz and
  M.~Yor,
  \emph{Continuous Martingales and Brownian Motion,
  Grundlehren der mathematischen Wissenschaften} (Springer, Berlin-Heidelberg, 1999).

\bibitem{Rogers2000}
L.~Rogers and
  D.~Williams,
  \emph{Diffusions, Markov Processes, and Martingales: Volume
  1, Foundations, Cambridge Mathematical Library}
  (Cambridge University Press, Cambridge, 2000).

\bibitem{Majumdar2004}
S.~N. Majumdar and
  A.~Comtet,
  Phys. Rev. Lett. \textbf{92},
  225501 (2004).
  

\bibitem{foot1}
The difference is that Brownian bridge is defined for a time continuous Brownian motion, but the equivalence to the discrete random walk problem can be demonstrated in the proper limit.

\bibitem{foot2}
In fact it is a Gaussian with a cut off on the tails of the distribution due to the fixed length of the polymer. This effect is analogous to the effect of the finite velocity of diffusing particles. It does not change the Gaussian nature of the bulk of the distribution and only affects its far tails.

\bibitem{foot3}
Interestingly, it can be shown that the statistics of rod orientations in a one-dimensional case is given by the Fermi-Dirac distribution. This problem will be discussed in detail elsewhere.

\bibitem{Greene2008}
W.~Greene,
  \emph{Econometric Analysis}
  (Prentice Hall, Upper Saddle River, NJ, 2008).

\bibitem{Athreya2006}
K.~Athreya and
  S.~Lahiri,
  \emph{Measure Theory and Probability Theory, Springer Texts
  in Statistics }(Springer, New York, 2006).

\bibitem{zickler1999meiotic}
D.~Zickler and
  N.~Kleckner,
  Annu. Rev. Genet. \textbf{33},
  603 (1999).

\bibitem{cromie2006single}
G.~A. Cromie,
  R.~W. Hyppa,
  A.~F. Taylor,
  K.~Zakharyevich,
  N.~Hunter, and
  G.~R. Smith,
  Cell \textbf{127}, 1167
  (2006).

\bibitem{marshall2001chromosome}
W.~F. Marshall,
  J.~F. Marko,
  D.~A. Agard, and
  J.~W. Sedat,
  Curr. Biol. \textbf{11},
  569 (2001).

\bibitem{alexander1991}
S.~P. Alexander
  and C.~L.
  Rieder, J. Cell Biol.
  \textbf{113}, 805 (1991).

\bibitem{kalinina2013pivoting}
I.~Kalinina,
  A.~Nandi,
  P.~Delivani,
  M.~R. Chac\'on,
  A.~H. Klemm,
  D.~Ramunno-Johnson,
  A.~Krull,
  B.~Lindner,
  N.~Pavin, and
  I.~M. Toli\'c-N{\o}rrelykke,
  Nat. Cell Biol. \textbf{15},
  82 (2013).

\bibitem{bystricky2004long}
K.~Bystricky,
  P.~Heun,
  L.~Gehlen,
  J.~Langowski,
  and S.~M.
  Gasser, Proc. Natl. Acad. Sci. U. S. A.
  \textbf{101}, 16495
  (2004).

\bibitem{cui2000pulling}
Y.~Cui and
  C.~Bustamante,
  Proc. Natl. Acad. Sci. U. S. A.
  \textbf{97}, 127 (2000).

\bibitem{langowski2006polymer}
J.~Langowski,
  Eur. Phys. J. E \textbf{19}, 241
  (2006).

\bibitem{rosa2008}
A.~Rosa and
  R.~Everaers,
  PLoS Comput. Biol. \textbf{4},
  e1000153 (2008).

\bibitem{ding2006meiotic}
D.-Q. Ding,
  N.~Sakurai,
  Y.~Katou,
  T.~Itoh,
  K.~Shirahige,
  T.~Haraguchi,
  and Y.~Hiraoka,
  J. Cell Biol.
  \textbf{174}, 499 (2006).

\bibitem{gennerich2007force}
A.~Gennerich,
  A.~P. Carter,
  S.~L. Reck-Peterson,
  and R.~D. Vale,
  Cell \textbf{131}, 952
  (2007).

\bibitem{toba2006overlapping}
S.~Toba,
  T.~M. Watanabe,
  L.~Yamaguchi-Okimoto,
  Y.~Y. Toyoshima,
  and H.~Higuchi,
  Proc. Natl. Acad. Sci. U. S. A.
  \textbf{103}, 5741 (2006).
  
 \bibitem{zhang2012}
 Y.~Zhang and D.~W.~Heermann, PLoS ONE \textbf{6}, e29225 (2012).
 
 \bibitem{zhang2013}
Y.~Zhang, S. Isbaner, and D.~W.~Heermann, Front. Phys. \textbf{1}, DOI=10.3389/fphy.2013.00016 (2013).
\end{thebibliography}

%\end{thebibliography}


\end{document}


