\documentclass[12pt,a4paper]{article}

\usepackage{color}
\usepackage{amsxtra}  
\usepackage{amsthm}
\usepackage{amssymb}
\usepackage{amsfonts}
\usepackage{graphicx}  
\usepackage{rotating}


\begin{document}

\title{Report: Rouse theory of pinned polymer loop}
\author{Wenwen Huang}
\date{\today}

\maketitle

\section{Model}
\label{sec:model}

Consider a pinned polymer loop modeled by bead-spring, the dynamical equation
for a single bead is 

\begin{equation}
    \label{eq:bead}
    \xi \frac{d \mathbf{r}_j}{dt} = - k_H \sum_{k} A_{jk} \mathbf{r}_k + \mathbf{f}_j^e + \mathbf{f}_j^b
\end{equation}
where $\xi$ is the friction coefficient of bead in solution, $\mathbf{r}_j$ is
the bead position of the $j$th bead, $k_H$ is the spring constant with a linear
Hookean spring assumed. $\mathbf{f}_j^e$ is the external force exerted on beads,
$\mathbf{f}_j^b$ is typical brownian force satisfying 
\begin{equation}
    \label{eq:brownian}
    \left<\mathbf{f}_j^b\right> = \mathbf 0;
    \left<f_{i\alpha}^b(t)f_{j\beta}^b(t^\prime)\right> = 2\xi k_B T \delta_{ij}
    \delta_{\alpha\beta}\delta(t-t^\prime)
\end{equation}
$\mathbf A$ is the connecting matrix, in case of pinned polymer loop, A has the form
\begin{equation}
    \label{eq:connectMatrix}
    \mathbf{A} = \begin{bmatrix}
        2 & -1 & 0   & \cdots   \\
        -1 & 2 & -1  &  \cdots  \\
        \vdots & \ddots &\ddots&\vdots\\
        \cdots & -1 &2 & -1 \\
        \cdots & 0 &-1 &2
    \end{bmatrix}
\end{equation}
In addition to the dynamical equation, the boundary condition for a pinned
polymer loop writes
\begin{equation}
    \label{eq:boudary}
    \mathbf r_0 = \mathbf r_N = 0
\end{equation}

\section{Normal Coordinate}
\label{sec:normalCoordinate}

Eq. \eqref{eq:bead} can be rewrite in the term of vector contains all beads
\begin{equation}
    \label{eq:beadVector}
    \xi \frac{d \mathbf{R}}{dt} = - k_H \mathbf{A} \mathbf{R} + \mathbf{F}^e + \mathbf{F}^b
\end{equation}
where $\mathbf{R} = \left[\mathbf r_1, \mathbf r_2, \cdots, \mathbf r_N\right]^T$,
and similar for $\mathbf{F}^e$, $\mathbf{F}^b$.

Notice that the connecting matrix $\mathbf{A}$ is real and symmetric, consider
a similarity transfer 
\begin{equation}
    \label{eq:similarityTransfer}
    \left[\Omega^{-1} A \Omega\right]_{jk} = \lambda_k\delta_{jk}
\end{equation}
here $\Omega$ is a unitary matrix, thus $\lambda_k$ is the eigenvalue of
matrix $\mathbf A$. 





\section{Constant Force Field}
\label{sec:constantForceField}
We now consider the pinned polymer loop in a constant force field, i.e.
$\mathbf{f}_j^e = f^e \mathbf{e}_z$.




\bibliography{report} \bibliographystyle{plain}

\end{document}
