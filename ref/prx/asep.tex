\documentclass[aps,showpacs,twocolumn,floatfix,prx,superscriptaddress]{revtex4-1}
\usepackage{graphicx}
\usepackage{amsfonts}
\usepackage{amsmath}
\usepackage{amssymb}
\usepackage{upgreek}
\usepackage[usenames,dvipsnames]{color}

%\bibliographystyle{apsrev}


\def\s{\sigma}
\begin{document}

\title{Maps to Particle Diffusion and Back in 1D}

\author{Wenwen Huang}
\author{Yen Ting Lin}
\author{Daniela Fr\"{o}mberg}
\author{Jaeoh Shin}
\author{Frank J\"{u}licher}
\author{Vasily Zaburdaev}
\affiliation{Max Planck Institute for the Physics of Complex Systems, N\"{o}thnitzer Str. 38, D-01187 Dresden, Germany}


\begin{abstract}
{}
\end{abstract}
\maketitle

\section{Asymmetric Exclusion Process}
Having shown the equilibrium statistics been solved by mapping from polymer to
particle, we now come to the discussion about the dynamics. It is intuitively to
extend the analogy to nonequilibrium, i.e., the dynamics of pinned polymer
corresponds to particle diffusion in a one dimensional lattice. To illustrate
the equivalence, we firstly define a typical particle hopping model and build
the connection between these two models. 

As shown in the section above, we consider a 1D lattice with $N$ lattice sites and
exact $N/2$ particles. Only simple exclusive interaction between particle is
applied, which means that one lattice site can only occupied by at most one
particle and the order of particles is conserved during the particle hopping
process. Denote the probability of particle hopping to right and left with $p$
and $q$ respectively, we have the following detailed balance during the hopping
\begin{equation}
    p P_{n} = q P_{n+1} \label{eq:db}
\end{equation}
where $P_{n}$ is the probability of configuration before particle hopping to
the right and $P_{n+1}$ is the probability of configuration after hopping. 
In addition, the ratio of of probability should be proportional to a Boltzmann
factor with the energy difference between these two configurations. Eq
\ref{eq:db} can be rewrite as 
\begin{equation}
    q / p = P_{n} / P_{n+1} = \exp{(-\Delta E / k_B T)}  \label{eq:p_divide_q}
\end{equation}

On the other hand, for a specific particle hopping system, the total hopping
rate is determined by the temperature. External force changes nothing but the
ratio $q/p$. Thus we have
\begin{equation}
    p + q = cT \label{eq:p_plus_q}
\end{equation}
where $c$ is a constant. With eq. \ref{eq:p_plus_q} and eq. \ref{eq:p_plus_q}
we can in principle solve $p$ and $q$ uniquely. The key quantity here is
$\Delta E$, which actually connects polymer and particle model. One can learn
from the polymer and particle equivalence that one particle hopping the right
corresponds to the change of two consecutive rods orientation from right-left to
left-right. Thus the energy difference of the two configuration writes
\begin{equation}
    \Delta E = 2F\Delta l
\end{equation}
where $F$ is the strength of external force and $\Delta l $ is the rod length.
Plug into the above equations one obtain
\begin{eqnarray}
    \label{eq:p_and_q}
    p & =  & \frac{cT\exp{(-2F\Delta l / k_B T)}}{1+\exp{(-2F\Delta l / k_B
            T)}} \\
    q & =  & \frac{cT}{1+\exp{(-2F\Delta l / k_B T)}}
\end{eqnarray}
Now we have a well defined particle hopping model equivalent to polymer
dynamics in the bulk, but the boundary condition is still not specified. It
turns out the boundary condition combined with particle number are crucial to
determinate the type of corresponding polymer.



\section{Towards understanding dynamics}

\section{Conclusions}

\begin{acknowledgments}
We would like to acknowledge stimulating discussions with M. Majumdar.\end{acknowledgments}


\begin{thebibliography}{10}


\bibitem{alberts2002}
B.~Alberts,
  A.~Johnson,
  J.~Lewis,
  M.~Raff,
  K.~Roberts, and
  P.~Walter,
  \emph{Molecular Biology of the Cell, Fourth Edition}
  (Garland Science, New York, 2002),
  4th ed.

\bibitem{gerton2005}
J.~L. Gerton and
  R.~S. Hawley,
  Nat. Rev. Genet. \textbf{6},
  477 (2005).

\bibitem{villeneuve2001}
A.~M. Villeneuve
  and K.~J.
  Hillers, Cell
  \textbf{106}, 647 (2001).

\bibitem{McKee2004}
B.~D. McKee,
  BBA-Gene. Struct. Expr. \textbf{1677}, 165 
  (2004).

\bibitem{Egel2004}
R.~Egel,
  \emph{The Molecular Biology of Schizosaccharomyces pombe:
  Genetics, Genomics and Beyond}(Springer, Berlin-Heidelberg,
  2004).

\bibitem{davis2001}
L.~Davis and
  G.~R. Smith,
  Proc. Natl. Acad. Sci. U. S. A.
  \textbf{98}, 8395 (2001).

\bibitem{munz1994}
P.~Munz,
  Genetics \textbf{137},
  701 (1994).

\bibitem{wells2006}
J.~L. Wells,
  D.~W. Pryce, and
  R.~J. McFarlane,
  Yeast \textbf{23}, 977
  (2006).
  
 \bibitem{SM} See Supplemental Material at ... for detailed description of experimental procedures, numerical simulations, and the discussion of 3D fluctuations of the polymer loop.

\bibitem{ding1998oscillatory}
D.-Q. Ding,
  Y.~Chikashige,
  T.~Haraguchi,
  and Y.~Hiraoka,
  J. Cell Sci. \textbf{111},
  701 (1998).

\bibitem{vogel2009self}
S.~K. Vogel,
  N.~Pavin,
  N.~Maghelli,
  F.~J\"ulicher,
  and I.~M.
  Toli\'c-N\o rrelykke, PLoS Biol.
  \textbf{7}, e1000087
  (2009).

\bibitem{yamamoto2001dynamic}
A.~Yamamoto,
  C.~Tsutsumi,
  H.~Kojima,
  K.~Oiwa, and
  Y.~Hiraoka,
  Mol. Biol. Cell \textbf{12},
  3933 (2001).

\bibitem{yamamoto1999cytoplasmic}
A.~Yamamoto,
  R.~R. West,
  J.~R. McIntosh,
  and Y.~Hiraoka,
  J. Cell Biol.
  \textbf{145}, 1233 (1999).

\bibitem{ananthanarayanan2013dynein}
V.~Ananthanarayanan,
  M.~Schattat,
  S.~K. Vogel,
  A.~Krull,
  N.~Pavin, and
  I.~M. Toli\'c-N\o rrelykke,
  Cell \textbf{153}, 1526
  (2013).

\bibitem{koszul2009dynamic}
R.~Koszul and
  N.~Kleckner,
  Trends Cell Biol. \textbf{19},
  716 (2009).

\bibitem{ding2004dynamics}
D.-Q. Ding,
  A.~Yamamoto,
  T.~Haraguchi,
  and Y.~Hiraoka,
  Dev. Cell \textbf{6},
  329 (2004).

\bibitem{wynne2012dynein}
D.~J. Wynne,
  O.~Rog,
  P.~M. Carlton,
  and A.~F.
  Dernburg, J. Cell Biol.
  \textbf{196}, 47 (2012).

\bibitem{Doi1986}
M.~Doi and
  S.~Edwards,
  \emph{The theory of polymer dynamics, International series
  of monographs on physics} (Clarendon Press, Oxford,
  1986).
  
\bibitem{foot0}
{\color{blue} Since the recombination machinery, locally altering the properties of the chromatin, becomes active only after the homologous chromosomes come to close proximity, we assume the effective temperature to be spatially uniform.}

\bibitem{Revuz1999}
D.~Revuz and
  M.~Yor,
  \emph{Continuous Martingales and Brownian Motion,
  Grundlehren der mathematischen Wissenschaften} (Springer, Berlin-Heidelberg, 1999).

\bibitem{Rogers2000}
L.~Rogers and
  D.~Williams,
  \emph{Diffusions, Markov Processes, and Martingales: Volume
  1, Foundations, Cambridge Mathematical Library}
  (Cambridge University Press, Cambridge, 2000).

\bibitem{Majumdar2004}
S.~N. Majumdar and
  A.~Comtet,
  Phys. Rev. Lett. \textbf{92},
  225501 (2004).
  

\bibitem{foot1}
The difference is that Brownian bridge is defined for a time continuous Brownian motion, but the equivalence to the discrete random walk problem can be demonstrated in the proper limit.

\bibitem{foot2}
In fact it is a Gaussian with a cut off on the tails of the distribution due to the fixed length of the polymer. This effect is analogous to the effect of the finite velocity of diffusing particles. It does not change the Gaussian nature of the bulk of the distribution and only affects its far tails.

\bibitem{foot3}
Interestingly, it can be shown that the statistics of rod orientations in a one-dimensional case is given by the Fermi-Dirac distribution. This problem will be discussed in detail elsewhere.

\bibitem{Greene2008}
W.~Greene,
  \emph{Econometric Analysis}
  (Prentice Hall, Upper Saddle River, NJ, 2008).

\bibitem{Athreya2006}
K.~Athreya and
  S.~Lahiri,
  \emph{Measure Theory and Probability Theory, Springer Texts
  in Statistics }(Springer, New York, 2006).

\bibitem{zickler1999meiotic}
D.~Zickler and
  N.~Kleckner,
  Annu. Rev. Genet. \textbf{33},
  603 (1999).

\bibitem{cromie2006single}
G.~A. Cromie,
  R.~W. Hyppa,
  A.~F. Taylor,
  K.~Zakharyevich,
  N.~Hunter, and
  G.~R. Smith,
  Cell \textbf{127}, 1167
  (2006).

\bibitem{marshall2001chromosome}
W.~F. Marshall,
  J.~F. Marko,
  D.~A. Agard, and
  J.~W. Sedat,
  Curr. Biol. \textbf{11},
  569 (2001).

\bibitem{alexander1991}
S.~P. Alexander
  and C.~L.
  Rieder, J. Cell Biol.
  \textbf{113}, 805 (1991).

\bibitem{kalinina2013pivoting}
I.~Kalinina,
  A.~Nandi,
  P.~Delivani,
  M.~R. Chac\'on,
  A.~H. Klemm,
  D.~Ramunno-Johnson,
  A.~Krull,
  B.~Lindner,
  N.~Pavin, and
  I.~M. Toli\'c-N{\o}rrelykke,
  Nat. Cell Biol. \textbf{15},
  82 (2013).

\bibitem{bystricky2004long}
K.~Bystricky,
  P.~Heun,
  L.~Gehlen,
  J.~Langowski,
  and S.~M.
  Gasser, Proc. Natl. Acad. Sci. U. S. A.
  \textbf{101}, 16495
  (2004).

\bibitem{cui2000pulling}
Y.~Cui and
  C.~Bustamante,
  Proc. Natl. Acad. Sci. U. S. A.
  \textbf{97}, 127 (2000).

\bibitem{langowski2006polymer}
J.~Langowski,
  Eur. Phys. J. E \textbf{19}, 241
  (2006).

\bibitem{rosa2008}
A.~Rosa and
  R.~Everaers,
  PLoS Comput. Biol. \textbf{4},
  e1000153 (2008).

\bibitem{ding2006meiotic}
D.-Q. Ding,
  N.~Sakurai,
  Y.~Katou,
  T.~Itoh,
  K.~Shirahige,
  T.~Haraguchi,
  and Y.~Hiraoka,
  J. Cell Biol.
  \textbf{174}, 499 (2006).

\bibitem{gennerich2007force}
A.~Gennerich,
  A.~P. Carter,
  S.~L. Reck-Peterson,
  and R.~D. Vale,
  Cell \textbf{131}, 952
  (2007).

\bibitem{toba2006overlapping}
S.~Toba,
  T.~M. Watanabe,
  L.~Yamaguchi-Okimoto,
  Y.~Y. Toyoshima,
  and H.~Higuchi,
  Proc. Natl. Acad. Sci. U. S. A.
  \textbf{103}, 5741 (2006).
  
 \bibitem{zhang2012}
 Y.~Zhang and D.~W.~Heermann, PLoS ONE \textbf{6}, e29225 (2012).
 
 \bibitem{zhang2013}
Y.~Zhang, S. Isbaner, and D.~W.~Heermann, Front. Phys. \textbf{1}, DOI=10.3389/fphy.2013.00016 (2013).
\end{thebibliography}

%\end{thebibliography}


\end{document}


