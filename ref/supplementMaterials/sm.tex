\documentclass{article}

\begin{document}

\section{Supplement Materials}


\subsection{Simulation}
\label{sub:simulation}
To control the number of free parameters, we choose the ideal bead-rod polymer model to describe the dynamics of chromosomes in nucleus, without considering the exclusive volume effect. 
As we will see later that the system can be reduced to a one free parameter model, i.e. $T_{eff}$.  

The simulation of model bead connected with rigid rod utilize the technique of Brownian Dynamics\cite{Cruz2012}.
The dynamical equation of beads representing chromosome loci is
\begin{equation}
	\label{eq:differential}
	\dot{\mathbf{r}_i} = \frac{1}{\xi}(\mathbf{F}_i^b + \mathbf{F}_i^c + \mathbf{F}_i^e + \mathbf{F}_i^{pseudo}) 
\end{equation}
where $\mathbf{r}_i$ is the position vector of the $i$th bead, $\xi$ is friction coefficient, $\mathbf{F}_i^b$ is random force, $\mathbf{F}_i^c$ is constraint force caused by rigid rod constraints, $\mathbf{F}_i^e$ is external force and $\mathbf{F}_i^{pseudo}$ is pseudo force added to mimic the statistics of bead-spring. 
Notice that the statistics of bead-rod is not exactly same as bead-spring. Subtle differences are caused by the intrinsic ``rigid'' of rods\cite{Hinch1994,Cruz2012}.

The random force, which characterize the fluctuation origin of beads dynamics, is a typical Brownian force satisfies the following conditions
\begin{equation}
	<\mathbf{F}_i^b> = \mathbf{0}; <\mathbf{F}_i^b(t)\mathbf{F}_j^b(t^{\prime})> = 2k_B T_{c} \xi \delta_{ij} \delta(t-t^{\prime})
\end{equation}
$k_B$ is Boltzmann constant and $T_{c}$ is \emph{characterizing temperature} characterize the level of randomness arise from the thermal motion of solvent molecules and some sorts of interactions between chromosome and proteins.

The constraint force for a specific bead in a bead-rod ring writes
\begin{equation}
	\mathbf{F}_i^c = \lambda_i \mathbf{u}_i - \lambda_{i-1} \mathbf{u}_{i-1}
\end{equation}
where $\lambda_i$ is strength of tension on the rod between $i$th and $(i+1)$th bead, $\mathbf{u}_i$ is the unit vector along this rod and $i=N$ ends to $i=0$.

In case of constant force field, the external force is constant $\mathbf{F}_i^e = -\xi \mathbf{v}$ acting on every bead except the pinned one representing SPB. 

The pseudo force is calculated using
\begin{equation}
	\mathbf{F}_i^{pseudo} = -\frac{\partial U_{met}}{\partial\mathbf{r}_i}; U_{met} = \frac{1}{2}k_B T_c \ln(\det G)
\end{equation}
where $G$ is the metric matrix of the bead-rod system\cite{Pasquali2002}.

Since the rods present in our model are rigid rods, additional constrained equations are needed to keep the rod length unchangeable
\begin{equation}
	\label{eq:constraint}
	(\mathbf{r}_{i+1} - \mathbf{r}_{i})^2 - a^2 = 0
\end{equation}
where $a$ is rod length.

Parameters above can be eliminated and dimensionless term of the dynamical equations can be obtained by the scaling $\mathbf{r}^{\prime}\to \mathbf{r}/a$; $t^{\prime}\to t/(\xi a^2/k_BT_c)$; $\mathbf{F}^{\prime}\to\mathbf{F}/(k_BT_c/a)$.
Our only free parameter \emph{effective temperature} which is also dimensionless is defined as
\begin{equation}
	\label{eq:Teff}
	T_{eff} = \frac{k_BT_c}{Fa} = \frac{k_B T_c}{\xi v a}
\end{equation}
Numerical scheme employed to solved the set of constrained differential equations (\ref{eq:differential}) and (\ref{eq:constraint}) is predictor-corrector algorithm used widely in bead-rod simulation\cite{Cruz2012,Somasi2002,Liu1989}.
Basic steps include calculating a prediction of $\mathbf{r}_i(t+\delta t)$ without considering the constraint force followed by a correction step, i.e.\ solving the algebra constraint equations to get constraint forces and re-plugin to the original equations for the corrected $\mathbf{r}_i(t+\delta t)$.
The differential equations are solved using Euler iterative method with a time step $dt = 10^{-4}$.  Statistical results are all obtained based on the ensemble of $10^{10}$ steps after equilibrium.  

\subsection{Parameter estimation}
\label{sub:estimation}
Experimentalists defined the pair of loci of homologous are "associated" for distance between them less than 0.35 $\mu m$\cite{Ding2004}. 
So we assume the threshold of the chromosome been aligned is 0.35 $\mu m$.

Then from what we have in our model, the distance between loci of different effective temperature follows Fig. 3b in the main text.
Apparently the distance varies with position along the chromosome. 
But let us consider the midpoint of the loop which has the largest divergence, hence the largest drag force is needed.
It turns out the corresponding effective temperature for the midpoint of a polymer loop with $N=500$ is $T_{eff}\approx 1$ (The exact value is around $1.16$). 
Actually $N$ does not matter too much given such a low temperature region. 

Substitute it into the definition of effective temperature Eq.~\ref{eq:Teff}, use the room temperature as characterizing temperature $T_c = 300K$ and Kuhn length $a = 100nm$, we obtain an estimation of the strength of force field we need.
\begin{equation}
	F = 3.57\times10^{-14} N
\end{equation}

We can further estimate the needed moving speed based on the simplest Stroke's law and sphere model. 
Assume the force field is generated from the friction of the bead sphere. 
The Stroke's law gives $F= 6\pi \mu R v$, where $\mu$ is the viscosity of the nucleoplasm, which is $1000$ folds of that of water.

$R$ is the radius of the bead sphere, which is hard to estimate given the complexity of real chromosome and the extremely simplified bead sphere model.
Nevertheless, people who use this sort of bead chain model estimated the diameter of the abstracted bead is around $30nm$\cite{Rosa2008}.

Use the above estimations of $R$, which would be a little bit underestimate, we obtain the moving velocity needed for the homologous to align is $v = 1.41\times10^{-7} m/s$ which is close to the experimental observation $v_{exp} = 4.17\times10^{-8}m/s$.
Another reasonable estimation $R=50nm$, which is half of the Kuhn length, gives $v = 4.24\times10^{-8}$, which is very close to the experimental observation.

\subsection{Parameter estimation for fission yeast}
\label{sub:estimation}

Fission yeast is a ideal model used very often in genetic studies. During meiosis, a fission yeast contains three pairs of chromosomes. The parameters measured from the experiments are summarized in Table~\ref{tab:parameters}.
\begin{table}[!ht]
	\caption{Parameters of fission yeast during meiosis}
	\label{tab:parameters}
	\begin{tabular}{l|l}
		\hline
		\textbf{Parameter} & \textbf{Value} \\
		\hline
		Typical size of nucleus          &  $3\mu m$ \\
		Chromosome number                &  Three pairs \\
		Compaction ratio of chromatin    &  $10^2bp/nm$ \\
                Kuhn length of chromatin         &  $100\sim300nm$  \\
		Duration of nuclear oscillation  &  $2 hours$  \\
		Period of nuclear oscillation    &  $10 min$  \\
		Moving speed of nucleus          &  $2.5\mu m/min$ \\
		Viscosity of nucleoplasm         &  $1000\times \mu_{water}$ \\
		\hline
	\end{tabular}

\end{table}

We use a bead-rod model to quantitatively characterize the dynamics of chromosomes of fission yeast. 
The question is how many beads are there that properly describe fission yeast genome. This is estimated by compacted contour length of chromosome divided by Kuhn length. 
The contour length of every chromatin can be exactly measured in base pair. Compacted contour length can be estimated when plugin the compaction ratio of chromatin.
The information of system size parameters are summarized in Table~\ref{tab:size}.
\begin{table}[!ht]
	\caption{Estimation of system size}
	\label{tab:size}
	\begin{tabular}{l|c|c}
		\hline
		\textbf{Chromosome} & \textbf{Length in base pair} & \textbf{Corresponding bead number} \\
		\hline
                Chromosome I   & $5.58Mb$  & $279$ \\
                Chromosome II  & $4.54Mb$  & $227$ \\
                Chromosome III & $2.45Mb$  & $123$ \\
		\hline
	\end{tabular}
\end{table}

\subsection{Fit model to fission yeast}
\label{sub:fit}
By setting parameters of our model accordingly to fission yeast, we can check the whether our theory fit the fission yeast system or not.
The homologous paring is accomplished during nuclear oscillation. However, the velocity of the nucleus is nearly constant when it moves from one end to the other end of the cell\cite{Vogel2009}.
Experimentalists defined the pair of loci of homologous are ``associated'' for distance between them less than $0.35\mu m$\cite{Ding2004}. Then we can check how strong the force field in our model required to align the chromosomes.

As in the experiments, we assume the threshold of the chromosome been aligned is $0.35\mu m$. Note that the length unit in our model corresponds to Kuhn length which is $100nm$.
Then from what we have in our model, the distance between loci of different effective temperature follows Fig. 3b in the main text.
Apparently the distance varies with position along the chromosome. 
But let us consider the midpoint of the loop which has the largest divergence, hence the largest drag force is needed.
It turns out the corresponding effective temperature for the midpoint of a polymer loop with $N=500$ is $\tilde{T}\approx 1$ (the exact value is around $1.16$). 
Actually $N$ does not matter too much given such a low temperature region. 

Substitute it into the definition of effective temperature Eq.~\ref{eq:Teff}, use the room temperature as characterizing temperature $T_c = 300K$, we obtain an estimation of the strength of force field we need.
\begin{equation}
	F = 3.57\times10^{-14} N
\end{equation}

Since the nuclear oscillation in fission yeast is driven by pulling force accumulated by dynein motors through the microtubules, the roughly number of motors needed can be estimated given the total bead number and stall force of each dynein motor.
It turns out the motor number needed to generate force field strength stated above is $13$, which is a reasonable and available number in fission yeast because the typical number involved in nuclear oscillation is about $50$ \cite{Vogel2009}.

Furthermore, we can estimate the needed moving speed for the alignment based on the simplest Stroke's law and sphere model. 
Assume the force field is generated from the friction of the bead sphere. 
The Stroke's law gives $F= 6\pi \mu R v$, where $\mu$ is the viscosity of the nucleoplasm, which is about $1000$ folds of that of water.

$R$ is the radius of the bead sphere, which is hard to estimate given the complexity of real chromosome and the extremely simplified bead sphere model.
Nevertheless, people who use this sort of bead chain model estimated the diameter of the abstracted bead is around $30nm$\cite{Rosa2008}.

Use the above estimations of $R$, which would be a little bit underestimate, we obtain the moving velocity needed for the homologous to align is $v = 1.41\times10^{-7} m/s$ which is close to the experimental observation $v_{exp} = 4.17\times10^{-8}m/s$.
Another reasonable estimation $R=50nm$, which is half of the Kuhn length, gives $v = 4.24\times10^{-8}$, which is very close to the experimental observation.

% \bibliography{sm}
\bibliographystyle{plain}

\begin{thebibliography}{1}

\bibitem{Cruz2012}
C.~Cruz, F.~Chinesta, and G.~R\'{e}gnier.
\newblock {Review on the Brownian Dynamics Simulation of Bead-Rod-Spring Models
  Encountered in Computational Rheology}.
\newblock {\em Archives of Computational Methods in Engineering},
  19(2):227--259, May 2012.

\bibitem{Ding2004}
DQ~Ding, Ayumu Yamamoto, Tokuko Haraguchi, and Yasushi Hiraoka.
\newblock {Dynamics of homologous chromosome pairing during meiotic prophase in
  fission yeast}.
\newblock {\em Developmental cell}, 6:329--341, 2004.

\bibitem{Hinch1994}
EJ~Hinch.
\newblock {Brownian motion with stiff bonds and rigid constraints}.
\newblock {\em Journal of Fluid Mechanics}, pages 219--234, 1994.

\bibitem{Liu1989}
Tony~W. Liu.
\newblock {Flexible polymer chain dynamics and rheological properties in steady
  flows}.
\newblock {\em The Journal of Chemical Physics}, 90(10):5826, 1989.

\bibitem{Pasquali2002}
Matteo Pasquali, David~C Morse, and Brownian Dynamics.
\newblock {An efficient algorithm for metric correction forces in simulations
  of linear polymers with constrained bond lengths}.
\newblock 116(5), 2002.

\bibitem{Rosa2008}
Angelo Rosa and Ralf Everaers.
\newblock {Structure and dynamics of interphase chromosomes.}
\newblock {\em PLoS computational biology}, 4(8):e1000153, January 2008.

\bibitem{Somasi2002}
Madan Somasi, Bamin Khomami, Nathanael~J. Woo, Joe~S. Hur, and Eric~S.G.
  Shaqfeh.
\newblock {Brownian dynamics simulations of bead-rod and bead-spring chains:
  numerical algorithms and coarse-graining issues}.
\newblock {\em Journal of Non-Newtonian Fluid Mechanics}, 108(1-3):227--255,
  December 2002.

\end{thebibliography}
\end{document}
