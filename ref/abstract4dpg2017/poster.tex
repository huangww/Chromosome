\documentclass[preprint, aps, superscriptaddress]{revtex4-1}
% \documentclass{article}

\begin{document}

\title{Quantification of polymer loop shapes in application to chromosome oscillations}

\author{Wenwen Huang}
\affiliation{Max Planck Institute for the Physics of Complex Systems, 01187, Dresden, Germany} 
\author{Vasily Zaburdaev}
\affiliation{Max Planck Institute for the Physics of Complex Systems, 01187, Dresden, Germany} 

\date{\today}

\begin{abstract}

    In this contribution, we model the chromosomes in meiotic fission yeast by pinned bead-rod loops in external force field. The 3D gyration tensor containing information of all beads positions is calculated. Based on the gyration tensor, the shape of polymer loops is quantified under different strength of the external force field. We show that the resulting shape is more rod-like and prolate under strong force and more sphere-like and oblate under weak force. Our study provides a quantitative description of the shape of pinned polymer loops under external field and may help us to describe the relevant biological processes in fission yeast such as chromosome oscillations and their alignment during meiosis. 

\end{abstract}
\maketitle

\end{document}

