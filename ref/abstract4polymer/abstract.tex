\documentclass{article}
\usepackage{authblk}

\begin{document}

\title{Brownian dynamics simulation of bead-rod rings models chromosome movements in fission yeast}

\author[a]{Wenwen Huang}
\author[a]{Yen Ting Lin}
\author[a]{Daniela Fr\"{o}mberg}
\author[a]{Frank J\"{u}licher}
\author[a,b]{Sergey Denisov}
\author[a]{Vasily Zaburdaev}\thanks{vzaburd@pks.mpg.de}
\affil[a]{Max Planck Institute for the Physics of Complex Systems, 01187, Dresden, Germany}
\affil[b]{Institute of Physics, University of Augsburg, Augsburg, Germany}
\date{\today}

\begin{abstract}

Bead-rod model are frequently used in both numerical and theoretical studies of polymers due to a fixed contour length and its intuitive properties. Brownian dynamics (BD) simulations is one of the most important methods in the study of bead-rod systems when applied to complex processes, such as DNA movement in a cell. Here, we present a numerical study of the ring polymer driven by an external force acting on a single bead. A system of several polymer loops linked and pulled at the same single spot models the process of chromosome arrangement during meiosis. Our simulations provide a quantitative way to investigate the contact dynamics of the homologous chromosomes and therefore better understand the physical mechanisms underlying the process of recombination. 

\end{abstract}
\maketitle

\end{document}

