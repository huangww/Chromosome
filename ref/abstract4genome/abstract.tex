\documentclass[preprint, aps, superscriptaddress]{revtex4-1}
% \documentclass{article}

\begin{document}

\title{Modeling meiotic chromosomes in fission yeast - from polymer conformation to single file diffusion and back}

\author{Wenwen Huang}
\affiliation{Max Planck Institute for the Physics of Complex Systems, 01187,
    Dresden, Germany} 
\author{Yen Ting Lin}
\affiliation{Max Planck Institute for the Physics of Complex Systems, 01187,
    Dresden, Germany} 
\affiliation{School of Physics and Astronomy, University of Manchester, M139PL,
    Manchester, United Kingdom.}
\author{Daniela Fr\"{o}mberg}
\affiliation{Max Planck Institute for the Physics of Complex Systems, 01187,
    Dresden, Germany} 
\author{Frank J\"{u}licher}
\affiliation{Max Planck Institute for the Physics of Complex Systems, 01187,
    Dresden, Germany} 
\author{Vasily Zaburdaev}
\affiliation{Max Planck Institute for the Physics of Complex Systems, 01187,
    Dresden, Germany} 

\date{\today}

\begin{abstract}
    In this contribution, we use a pinned loop polymer model to describe meiotic
    chromosomes in fission yeast. We show that the problem of finding the
    conformations of a pinned polymer loop in the external force field can be
    mapped to the corresponding problem of the single file diffusion. It allows
    us to find the exact solution for the equilibrium statistics of both
    systems, which turns out to be described by the Fermi-Dirac distribution.
    Moreover we can quantify not only the behavior of average positions of
    diffusing particles and monomers of the polymer loop but also their
    fluctuations. Fluctuations are affected by the constrains of the system and
    explicitly depend on the position. To close the loop of analogies we show
    that the kinetic Monte Carlo simulations, which can be performed for the
    single file diffusion with a well defined physical time, can be used to
    quantify the non-equilibrium dynamics of polymer loops.


\end{abstract}
\maketitle

\end{document}

