\documentclass[portrait,a0]{anjaposter}
\usepackage[utf8]{inputenc}

\usepackage{amsmath}

\title{synchronization in large-scale complex networks: exploiting the dynamical aspects}
\author{Li Chen}
\institute{Max-Planck-Institut für Physik komplexer Systeme, Dresden}
\email{chenli@pks.mpg.de}
\leftlogo{minerva-hhr.eps}
\rightlogo{logoblau-correct.eps}

\begin{document}

\background

\begin{poster}

\vspace{1cm}

\begin{pcolumn}{0.475}

\vfill

 \tbox{Collaborators}{
  \vspace*{-0.9cm}
  \begin{itemize}
   \item Can Qiu and Hong Bin Huang (Southeast Universtiy, PR China).
   \item Hai Jun Wang (Nanjing Xiaozhuang University, PR China)
   \item Guan Xiao Qi (Reaserch Center J\"ulich, Germany)
  \end{itemize}
}

 \tbox{Introduction}{
   \begin{itemize}
   \item Synchronization of large-scale complex networks is the functional basis of some real systems, such as brains, and also usually the aim of some engineering designs.
   \item To achieve this goal, the prevailing paradigm is to modify the network structure, such as to break the hub node into a group of subnodes, to make networks weighted or directed, or to compensate with negative links etc. However, the structure of networks in real-world cases are well developed not allowing major structural modifications.
    \item Here we attack this problem in another direction, i.e., instead of structure manipulations, by considering some factors in dynamical aspects the stable synchronization of large networks can also be facilitated considerably.
    \end{itemize}
} 
\vfill \tbox{Framework: Master stability function}{ Consider a
network of N oscillator with arbitrary complex network with
\begin{equation*}
\dot{x}_i=F(x_i)+\varepsilon(t)\sum_{j=1}^{N}G_{ij}H(x_j(\tau)),
\end{equation*}
where $G=(G_{ij})$ describes the topology of the complex networks,
and $\varepsilon$ and $\tau$ are coupling strength and time delay,
respectively (in the most studies, $\varepsilon$ is time constanst and no time delay). By linearization of the equations, the 
synchronous solution of the coupled systems is stable if the following 
relationship is satisfied
\begin{equation*}
R\equiv\frac{\lambda_N}{\lambda_2}<\frac{\alpha_2}{\alpha_1}\equiv S.
\end{equation*}
Here, $R$ is the eigenratio characterizing the synchronizability of
complex network, and $S$ is the threshold of master stability
function (MSF) ratio, only determined by dynamical aspects,
including the oscillator dynamics, coupling scheme, time delay etc.
Within this framework, most of previous studies try to reduce the
value of $R$ by modifying networks. Here, instead, we try to increase $S$,
while $R$ is kept unchange, to make the above relationship being better
satisfied. } 
\vfill \tbox{A simple on-off coupling}{
  \begin{itemize}
  \item The coupled chaotic R\"{o}ssler oscillators
 % \vspace*{-1cm}

   \begin{pminipage}{0.45}
      \myfig{onoff1.eps}{1.0}
   \end{pminipage}
   \begin{pminipage}{0.45}
      \myfig{onoff2.eps}{1.0}
   \end{pminipage}
   \mycaption{Master stability function as a function of coupling strength $\alpha$ and on-off ratio $\theta$ with on-off period T=2s, 3s, 6s, 9s for (a)-(d) (Left). The corresponding threshold ratio $S$, time step=0.01 (Right). 
   Note that in the limit of time step$\rightarrow 0$, $S$ goes to infinity.}
   \item An example within foodweb networks

   \begin{pminipage}{0.45}
\center
      \myfig{onoff3.eps}{1.3}
   \end{pminipage}
   \mycaption{Time evolution of synchronization error $\delta$ and time series (inset). Parameter $T=1$, $\theta=0.1$, $\varepsilon=100$. The results means that intensive interaction among commnunities in animal's active season for just few months can synchroniz e very large spatial zone. Notice that, for traditional constant coupling, there is a size instability for this coupled systems, while on-off coupling (accounted for seasonality) can avoid this.}
  \end{itemize}
}
\vfill
\vfill
 \tbox{References}{
    \vspace*{-0.25cm}

    \small
    [1] A. Arenas, et. al, (2008), {\sl Phys.\ Rep.} {\bf 469}, 93.

    [2] L. Chen, C. Qiu, and H. B. Huang, (2009), {\sl Phys.\ Rev.\ E} {\bf 79}, 045101(R).

   [3] L. Chen, C. Qiu, and H. B. Huang, G. X. Qi, and H. J. Wang (2010), {\sl Eur.\ Phys.\ J.\ B} {\bf 76}, 625.

    [4] L. Chen, C. Qiu, and H. B. Huang, G. X. Qi, and H. J. Wang (2010), {\sl Phys.\ Rev.\ E} {\bf 82}, 056115.
}
\vspace*{-4cm}

\end{pcolumn}
\hfill
\begin{pcolumn}{0.475}
\tbox{An activity-regulated coupling}{
  \begin{itemize}
  \item Dynamic coupling strength
\begin{equation*}
\varepsilon_{ij}=\varepsilon_0S(\frac{\Delta w_{ij}}{\Delta w'_{ij}}-l),
\end{equation*}
where $\Delta w_{ij}$ is the difference of the variable that used in
coupling between nodes $i$ and $j$, and $\Delta w'_{ij}$
corresponding to the rest; $l$ is the threhold that control coupling switching on/off. $S(x)=1/(1+e(-\alpha x))$ is a sigmoid
function.
   \item This dynamic coupling can be considered generalized on-off coupling (the sigmoid function becomes Heaviside function as $\alpha$ approaches positive infinity), and can apply to a more variety of chaotic oscillators.
   \item Notice that, this form is now beyond the MSF framework because each coupling strength is determined by the two node state, thus not uniform.

   \item An example is given as follows:

   \begin{pminipage}{0.45}
      \myfig{onoff5.eps}{1.0}
   \end{pminipage}
   \begin{pminipage}{0.45}
      \myfig{onoff4.eps}{1.0}
   \end{pminipage}
   \mycaption{(a) The transversal Lyapunov exponent as function of $\varepsilon_0$ for $z-$coupled Lorenz oscillators.(b) Time series for $\varepsilon_0=10$ in noisy backgroud ($\sim 10^{-5}$). upper panel: constant coupling; lower pannel: dynamic coupling leading to the symmetry broken for Lorenz equation $(-x,-y,z)\rightarrow(x,y,z)$ gets stable synchronization.
   The threshold $l=2$.}
      
  \item Self-organized functional networks in synchronization process with this dynamic coupling. Physically, the network is globally connected, and the functional connectivity between to two nodes exist if 
    $\varepsilon_{ij}>\varepsilon_c$ (e.g., $0.01\varepsilon_0$), vice versa.

   \begin{pminipage}{0.45}
      \myfig{4a.eps}{1.0}
   \end{pminipage}
   \begin{pminipage}{0.45}
      \myfig{4b.eps}{1.0}
   \end{pminipage}
   \mycaption{(a) Degree distribution for the self-organized functional networks, with a tail $p(k)\sim k^{-3.1}$. Data from networks of size $N=1000$ over $100$ realizations. (b) An examplified network with $N=100$.}

  \end{itemize}
}
\vfill

\tbox{Uniform time-delay coupled systems}{
  \begin{itemize}
  \item In this case, there is a uniform time delay in the the transmission of signal, as well as in their own signal.The resulting synchronous manifold remains the same as for the isolated individual.
  \item By including intermediate amount of time-delay, the stable region emerges. This comes from the phase structure of coupled dynamics.
  \item Interestingly, Fig. 5(b) shows that for this time-delayed network stable synchronization can survive in "all negative links".

   \begin{pminipage}{0.45}
      \myfig{delay1.eps}{1.0}
   \end{pminipage}
   \begin{pminipage}{0.45}
      \myfig{delay2.eps}{1.0}
   \end{pminipage}
   \mycaption{Master stability function as function of coupling strength $\alpha$ and uniform time delay $\tau$ for the nondiagnal coupling $x\rightarrow y$ (a) and $y\rightarrow x$ (b) respectively.}
  \end{itemize}
}
\vfill

 \tbox{Discussion and Conclusion}{
  \vspace*{-0.9cm}
  \begin{itemize}
%   \item Anomaly time series has some stationarity issues.
   \item By including some simple dynamical factors, such as dynamic
   coupling and time delay, the networked system allows better
   synchronization performance. Thus, the study opens a new view to
   consider the problem of synchronization in networked system.
   \item Meanwhile, we have to admit that the uniform on-off and uniform time delay adopted here may not be so realistic in real
   world. A more suitable treatment should be under the framework of
   adaptive networks.
   \item The research emphasizes the role of exploiting the dynamical
    structure of coupled units, which could actually be as much
    important as the role of structure part played in network
    synchronization.
  \end{itemize}
}
\vfill
\vspace*{-4cm}
\end{pcolumn}

\end{poster}
\end{document}
