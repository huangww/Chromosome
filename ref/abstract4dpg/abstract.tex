\documentclass[preprint,aps,draft]{revtex4}

\begin{document}

\title{Modeling chromosomes during meiosis in fission yeast using pinned polymer rings}

\author{Wenwen Huang}
\author{Yen Ting Lin}
\author{Daniela Fr\"{o}mberg}
\author{Frank J\"{u}licher}
\author{Vasily Zaburdaev}
\affiliation{Max Planck Institute for the Physics of Complex Systems, 01187, Dresden, Germany}
\affiliation{Max Planck Institute for the Physics of Complex Systems, 01187, Dresden, Germany}

\date{\today}

\begin{abstract}
Ring structured molecular has been discovered are ubiquitous among biological processes and play an important role, e.g., circular RNA controlling transcription. 
During the prophase of meiosis I in fission yeast, ends of chromatins are bonded to the Spindle Pole Body (SPB), form a ring structure. 
Furthermore, the whole nucleus are oscillating back and forth in the cylinder shape fission yeast cell, driven by the SPB through a pulling force along the micro tube.
The dramatic movements of chromosomes are believed to relate to chromatin alignment and recombination, help to proper separation of chromosome in later stage.
Using the pinned polymer ring model, we investigate the statistics of the loci position along chromatins.
Based on the concept of Brownian bridge, we derived the analytical result of distance between homologous pairs under certain noise level.
Furthermore, we verified our theoretical results using the Brownian dynamics simulation based on a bead-rod polymer model.
Our analytically solvable model helps us to understand the exact role the chromosome movements and is also applicable to other ring structured biological and soft mater cases.

	

\end{abstract}
\maketitle

\end{document}

