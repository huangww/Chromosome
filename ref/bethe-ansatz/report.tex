\documentclass[12pt,a4paper]{article}
\usepackage{color}
\usepackage{amsmath}  
\usepackage{amsthm}
\usepackage{amssymb}
\usepackage{amsfonts}
\usepackage{graphicx}  


\begin{document}

\title{Bethe Ansatz Solution of ASEP with Reflecting Boundaries}
\author{Wenwen Huang}
\date{\today}
\maketitle

To investigate the ASEP model of $N$ particles on $L$ lattice sites with
reflecting boundaries. Let us start with one single particle in such a closed
lattice. The master equation of the particle can be written as
\begin{subequations}
\begin{eqnarray}
    \label{eq:single-particle-a}
    \frac{d}{dt} P(x, t) & = & \alpha P(x-1, t) + \beta P(x+1, t) - (\alpha +
    \beta)P(x, t) \\
    \label{eq:single-particle-b}
    \frac{d}{dt} P(1, t) & = & \beta P(2, t) - \alpha P(1, t) \\
    \label{eq:single-particle-c}
    \frac{d}{dt} P(L, t) & = & \alpha P(L-1, t) - \beta P(L, t)
\end{eqnarray}
\end{subequations}
where $\alpha$ and $\beta$ is hopping rate of particle to left and right,
respectively. $x$ denotes the position of the particle and Eq.
\eqref{eq:single-particle-b} and \eqref{eq:single-particle-c} are actually the
special cases of master equation at the boundaries.  By assuming Eq.
\eqref{eq:single-particle-a} is valid for the whole space, we can rewrite
Eq. \eqref{eq:single-particle-b} and \eqref{eq:single-particle-c} as boundaries
condition 
\begin{subequations}
    \label{eq:boundaries-single-particle}
\begin{eqnarray}
    \alpha P(0, t) = \beta P(1, t) \\
    \alpha P(L, t) = \beta P(L+1, t)
\end{eqnarray}
\end{subequations}
The above equations use the technique so called ``ghost coordinate", i.e.
$x=0,L+1$. But physically they are essentially the same as the master equation
\eqref{eq:single-particle-b} and \eqref{eq:single-particle-c}, which means the
flux of particle are balanced in both direction, thus the reflecting boundaries.
The advantage of using Eq. \eqref{eq:boundaries-single-particle} is that it
simplifies the calculation a lot.  

To solve the case of single particle ASEP, we take the ansatz of separation of
variables $P(x, t) = \phi(x)e^{\lambda t}$ and plug into the master equation,
obtaining 
\begin{equation}
    \label{eq:eigen}
    \beta\phi(x+1) -(\alpha+\beta+\lambda)\phi(x) + \alpha\phi(x-1) = 0
\end{equation}
Given that $x$ is an integer number, Eq. \eqref{eq:eigen} is essentially a set
of liner difference equations with the boundaries by substituting the ansatz of
$P(x,t)$ into boundaries of Eq. \eqref{eq:boundaries-single-particle}: 
\begin{subequations}
    \label{eq:boundaries-phi}
\begin{eqnarray}
    \alpha\phi(0) = \beta\phi(1) \\
    \alpha \phi(L) = \beta \phi(L+1)
\end{eqnarray}
\end{subequations}
The standard method to find the solution is again to take an ansatz $\phi(x) =
Az^x$, where $z$ is an arbitrary complex number. We arrive at the characteristic
quadratic equation 
\begin{equation}
    \label{eq:characteristic}
    \beta z^2 - (\alpha + \beta + \lambda ) z + \alpha = 0
\end{equation}
The two roots fulfill $z_1z_2 = \frac{\alpha}{\beta}$. And the solution of
\eqref{eq:eigen} can be written as 
\begin{equation}
    \label{eq:eigen-solution}
    \phi(x) = C_1 z_1^x + C_2 z_2^x
\end{equation}
By applying the boundaries Eq. \eqref{eq:boundaries-phi} to Eq.
\eqref{eq:eigen-solution} we can find all the eigenvalues and corresponding
eigenvectors. The results are summarised as following
\begin{subequations}
    \label{eq:single-particle-eigenvalues}
    \begin{eqnarray}
        \lambda_k & = &
        \begin{cases}
            -(\alpha+\beta) + 2\sqrt{\alpha\beta}\cos(\frac{k\pi}{L});
            & k=1,2,\dots, L-1 \\
            0; & k=L
        \end{cases}
    \end{eqnarray}
    \label{eq:single-particle-eigenvectors}
    \begin{eqnarray}
        \phi_k(x) & = & 
        \begin{cases}
            const. \left(\frac{\alpha}{\beta}\right)^{\frac{x}{2}}
            \left[\sin\left(\frac{k\pi}{L}x\right) -
                \sqrt{\frac{\beta}{\alpha}}\sin\left(\frac{k\pi}{L}(x-1)\right)\right];
            & k = 1,2, \dots, L-1 \\
            const. \left(\frac{\alpha}{\beta}\right)^{x}; & k=L
        \end{cases}
    \end{eqnarray}
\end{subequations}
The eigenvalue $\lambda_L = 0$ and corresponding eigenvector represent the
stationary mode. By properly choose the constant, one can check the
orthogonality and completeness of the eigenfunctions. 
\begin{eqnarray}
    \label{eq:orthogonality}
    \sum_{x} \phi_k(x)\phi_l(x) & = & \delta_{k,l} \\
    \label{eq:completeness}
    \sum_{k} \phi_k(x)\phi_k(y) & = & \delta_{x,y} 
\end{eqnarray}
So for arbitrary initial distribution of $P(x, 0)$, we can always decompose it as 
\begin{equation}
    \label{eq:decompose-intial-single}
    P(x,0) = \sum_k{c_k \phi_k(x)}
\end{equation}
where $c_k$ can be calculated by 
\begin{equation}
    \label{eq:coeff-k}
    c_k = \sum_x{\phi_k(x) P(x,0)}
\end{equation}
Finally, the solution of single particle on reflecting lattice can be written as
\begin{equation}
    \label{eq:solution-single}
    P(x,t) = \sum_k{\phi_k(x)e^{\lambda_k t}}\sum_y{\phi_k(y)P(y,0)}
\end{equation}
For the special case that $P(x,0) = \delta_{x,y}$, solution
\eqref{eq:solution-single} can be simplified to
\begin{equation}
    \label{eq:solution-single-simplified}
    P(x,t) = \sum_k{\phi_k(x)\phi_k(y)e^{\lambda_k t}}
\end{equation}

With the complete solution of single particle, we can go further to systems of
more than one particle. For the purpose of illustrating the path of solution
searching, we go just one step to show the solution of two particles on $L$
lattice sites. The position of particles are denotes by $x_1, x_2$ with
constraint $x_1<x_2$.





% \bibliography{report}
\bibliographystyle{plain} 

\end{document}
