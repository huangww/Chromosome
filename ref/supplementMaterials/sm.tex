\documentclass{article}

\begin{document}

\section{Supplement Materials}

\subsection{Simulation}
\label{sub:simulation}
To control the number of free parameters, we choose the ideal bead-rod polymer model to describe the dynamics of chromosomes in nucleus, without considering the exclusive volume effect. 
As we will see later that the system can be reduced to a one free parameter model, i.e. $T_{eff}$.  

The simulation of model bead connected with rigid rod utilize the technique of Brownian Dynamics\cite{Cruz2012}.
The dynamical equation of beads representing chromosome loci is
\begin{equation}
	\label{eq:differential}
	\dot{\mathbf{r}_i} = \frac{1}{\xi}(\mathbf{F}_i^b + \mathbf{F}_i^c + \mathbf{F}_i^e + \mathbf{F}_i^{pseudo}) 
\end{equation}
where $\mathbf{r}_i$ is the position vector of the $i$th bead, $\xi$ is friction coefficient, $\mathbf{F}_i^b$ is random force, $\mathbf{F}_i^c$ is constraint force caused by rigid rod constraints, $\mathbf{F}_i^e$ is external force and $\mathbf{F}_i^{pseudo}$ is pseudo force added to mimic the statistics of bead-spring. 
Notice that the statistics of bead-rod is not exactly same as bead-spring. Subtle differences are caused by the intrinsic "rigid" of rods\cite{Hinch1994,Cruz2012}.

The random force, which characterize the fluctuation origin of beads dynamics, is a typical Brownian force satisfies the following conditions
\begin{equation}
	<\mathbf{F}_i^b> = \mathbf{0}; <\mathbf{F}_i^b(t)\mathbf{F}_j^b(t^{\prime})> = 2k_B T_{c} \xi \delta_{ij} \delta(t-t^{\prime})
\end{equation}
$k_B$ is Boltzmann constant and $T_{c}$ is \emph{characterizing temperature} characterize the level of randomness arise from the thermal motion of solvent molecules and some sorts of interactions between chromosome and proteins.

The constraint force for a specific bead in a bead-rod ring writes
\begin{equation}
	\mathbf{F}_i^c = \lambda_i \mathbf{u}_i - \lambda_{i-1} \mathbf{u}_{i-1}
\end{equation}
where $\lambda_i$ is strength of tension on the rod between $i$th and $(i+1)$th bead, $\mathbf{u}_i$ is the unit vector along this rod and $i=N$ ends to $i=0$.

In case of constant force field, the external force is constant $\mathbf{F}_i^e = -\xi \mathbf{v}$ acting on every bead except the pinned one representing SPB. 

The pseudo force is calculated using
\begin{equation}
	\mathbf{F}_i^{pseudo} = -\frac{\partial U_{met}}{\partial\mathbf{r}_i}; U_{met} = \frac{1}{2}k_B T_c ln(det G)
\end{equation}
where $G$ is the metric matrix of the bead-rod system\cite{Pasquali2002}.

Since the rods present in our model are rigid rods, additional constrained equations are needed to keep the rod length unchangeable
\begin{equation}
	\label{eq:constraint}
	(\mathbf{r}_{i+1} - \mathbf{r}_{i})^2 - a^2 = 0
\end{equation}
where $a$ is rod length.

Parameters above can be eliminated and dimensionless term of the dynamical equations can be obtained by the scaling $\mathbf{r}^{\prime}\to \mathbf{r}/a$; $t^{\prime}\to t/(\xi a^2/k_BT_c)$; $\mathbf{F}^{\prime}\to\mathbf{F}/(k_BT_c/a)$.
Our only free parameter \emph{effective temperature} which is also dimensionless is defined as
\begin{equation}
	T_{eff} = \frac{Fa}{k_BT_c} = \frac{\xi v a}{k_B T_c}
\end{equation}
Numerical scheme employed to solved the set of constrained differential equations (\ref{eq:differential}) and (\ref{eq:constraint}) is predictor-corrector algorithm used widely in bead-rod simulation\cite{Cruz2012,Somasi2002,Liu1989}.
Basic steps include calculating a prediction of $\mathbf{r}_i(t+\delta t)$ without considering the constraint force followed by a correction step, i.e.\ solving the algebra constraint equations to get constraint forces and re-plugin to the original equations for the corrected $\mathbf{r}_i(t+\delta t)$.
The differential equations are solved using Euler iterative method with a time step $dt = 10^{-4}$.  Statistical results are all obtained based on the ensemble of $10^{10}$ steps after equilibrium.  

% \bibliography{sm}
% \bibliographystyle{plain}
	

\begin{thebibliography}{1}

\bibitem{Cruz2012}
C.~Cruz, F.~Chinesta, and G.~R\'{e}gnier.
\newblock {Review on the Brownian Dynamics Simulation of Bead-Rod-Spring Models
  Encountered in Computational Rheology}.
\newblock {\em Archives of Computational Methods in Engineering},
  19(2):227--259, May 2012.

\bibitem{Hinch1994}
EJ~Hinch.
\newblock {Brownian motion with stiff bonds and rigid constraints}.
\newblock {\em Journal of Fluid Mechanics}, pages 219--234, 1994.

\bibitem{Liu1989}
Tony~W. Liu.
\newblock {Flexible polymer chain dynamics and rheological properties in steady
  flows}.
\newblock {\em The Journal of Chemical Physics}, 90(10):5826, 1989.

\bibitem{Pasquali2002}
Matteo Pasquali, David~C Morse, and Brownian Dynamics.
\newblock {An efficient algorithm for metric correction forces in simulations
  of linear polymers with constrained bond lengths}.
\newblock 116(5), 2002.

\bibitem{Somasi2002}
Madan Somasi, Bamin Khomami, Nathanael~J. Woo, Joe~S. Hur, and Eric~S.G.
  Shaqfeh.
\newblock {Brownian dynamics simulations of bead-rod and bead-spring chains:
  numerical algorithms and coarse-graining issues}.
\newblock {\em Journal of Non-Newtonian Fluid Mechanics}, 108(1-3):227--255,
  December 2002.

\end{thebibliography}

\end{document}
